% arara: xelatex
% arara: xelatex
% arara: xelatex


% options:
% thesis=B bachelor's thesis
% thesis=M master's thesis
% czech thesis in Czech language
% english thesis in English language
% hidelinks remove colour boxes around hyperlinks

\documentclass[thesis=B,english]{FITthesis}[2012/10/20]

\usepackage[utf8]{inputenc} % LaTeX source encoded as UTF-8
% \usepackage[latin2]{inputenc} % LaTeX source encoded as ISO-8859-2
% \usepackage[cp1250]{inputenc} % LaTeX source encoded as Windows-1250

\usepackage{graphicx} %graphics files inclusion
% \usepackage{subfig} %subfigures
% \usepackage{amsmath} %advanced maths
% \usepackage{amssymb} %additional math symbols

\usepackage{dirtree} %directory tree visualisation

\usepackage{csquotes}

\usepackage{todonotes}

%\usepackage[main=english,czech]{babel}
   
% % list of acronyms
% \usepackage[acronym,nonumberlist,toc,numberedsection=autolabel]{glossaries}
% \iflanguage{czech}{\renewcommand*{\acronymname}{Seznam pou{\v z}it{\' y}ch zkratek}}{}
% \makeglossaries

% % % % % % % % % % % % % % % % % % % % % % % % % % % % % % 
% EDIT THIS
% % % % % % % % % % % % % % % % % % % % % % % % % % % % % % 

\department{Department of Computer Systems}
\title{ Memory efficient cluster representations in non-metric spaces}
\authorGN{Jaroslav} %author's given name/names
\authorFN{Hlaváč} %author's surname
\author{Jaroslav Hlaváč} %author's name without academic degrees
\authorWithDegrees{Jaroslav Hlaváč} %author's name with academic degrees
\supervisor{Ing. Martin Kopp}
\acknowledgements{THANKS (remove entirely in case you do not with to thank anyone)} %TODO
\abstractEN{Summarize the contents and contribution of your work in a few sentences in English language.}
\abstractCS{V n{\v e}kolika v{\v e}t{\' a}ch shr{\v n}te obsah a p{\v r}{\' i}nos t{\' e}to pr{\' a}ce v {\v c}esk{\' e}m jazyce.} %TODO
\placeForDeclarationOfAuthenticity{Prague}
\keywordsCS{Replace with comma-separated list of keywords in Czech.} %TODO
\keywordsEN{Replace with comma-separated list of keywords in English.} %TODO
\declarationOfAuthenticityOption{1} %select as appropriate, according to the desired license (integer 1-6)
% \website{http://site.example/thesis} %optional thesis URL


\begin{document}

% \newacronym{CVUT}{{\v C}VUT}{{\v C}esk{\' e} vysok{\' e} u{\v c}en{\' i} technick{\' e} v Praze}
% \newacronym{FIT}{FIT}{Fakulta informa{\v c}n{\' i}ch technologi{\' i}}

\setsecnumdepth{part}
\chapter{Introduction}

Even after several decades after the Internet became the most used medium for communication, there are still many issues concerning its security. Businesses all around the world daily connect more of their infrastructure to the Internet thus creating a possibility for attackers to exploit the connected devices for their malicious use. Security engineers are always inventing new methods to stop the attacks and secure their networks. Attackers race to overcome these newly created methods and to gain control of protected data or devices. This race creates a need for much more sophisticated defense mechanisms. \\

Multi-layer defense systems are protecting contemporary networks. As attackers overcome one layer, there is another layer waiting for them after it. However, if the succeed to bypass all of them unnoticed a colossal problem can occur. They can collect user credentials,  exfiltrate data or gain control of devices without anybody's knowledge. It can take a long time before the breach is noticed. A similar thing happened in Marriot International child company Starwood last year. Names, emails, addresses and credit card numbers were stolen in an attack that was discovered in 2018 but possibly could have begun as early as in 2014\cite{hron2018breaches}. \\
                                                                                                                                                                    
One of several methods that are used to detect such intrusions is detecting anomalies in the behavior of network hosts - Network Behavioral Anomaly Detection (NBAD). NBAD compares the current behavior of the network to a model created from previous behavior. Devices inside of the network are used as probes to gather information about the behavior of the network or individual hosts. Models are calculated from the data collected. If a significant change in behavior is detected, security personnel are notified, and they can act accordingly. \\
                                                                                  
It is normal for a company to have up to millions of devices. Straight forward usage of NBAD in these big networks is not sufficient to detect breaches successfully. For example, a single infected device that can be easily missed in the traffic of a bigger company. In order to detect such anomalies in this network, it is needed to focus on smaller parts of it. Clustering based on the behavior of network hosts can be used to separate the network into smaller pieces. \\

Cognitive targeted anomaly detection framework from Cisco serves as an example of this clustering. Each cluster that it creates consists of hosts that behave similarly.  The main focus of this work is finding a representation of these clusters that can be used to calculate the model behavior and also be dynamically updated as the behavior of the network changes. \\ 
                                           
Four algorithms previously used to represent clusters in computer vision or other fields were studied and modified for usage in Cognitive anomaly detection framework. They were tested and compared on both real and test datasets. \\

(Chapters and sections will be briefly explained here.)

%The main focus of this work is in representing these clusters found by clustering algorithms.
%Finding representation of clusters in Cognitive targeted anomaly detection framework is made more difficult by the fact, that similarity measure used for clustering does not form a metric space.
%Therefore the focus on methods that can be used for selecting representatives in non-metric spaces. \\
%                                                                                  
%Each cluster can have up to several thousands of hosts.                              
%Algorithms considered in this thesis were selected based on the following expectations.
%As the hosts are clustered based on their behavior, it is safe to expect that selecting a subset of the cluster can represent its behavior without losing significant information.
%Furthermore having fewer hosts as representatives of each cluster significantly reduce memory usage. \\
                                                                                    
%Why use clustering methods in network anomaly detection?
%Why do we need representative selection?
%What are the next sections about?
%
%Automatically grouping similar objects from a given set into clusters is a
%widely used machine learning technique in many fields ranging from
%recommendation engines and social network analysis to image segmentation and
%medical imaging. New applications of this powerfull method are found every year.
%Clustering is also used for Anomaly Detection in computer systems. \par
%In this work I collected 5TODO known methods used in different fields for
%cluster representation. What these methods have in common is possibility to use
%them in spaces where only generic similarity measure exists (these spaces do 
%not necessarily need to be metric).

\chapter{Goals}

This work aims to study methods used for the representation of network host behavioral clusters that can be used in Cognitive targeted anomaly detection framework.
Behavioral similarity used for clustering in this framework does form a metric space, which creates a requirement for algorithms used for representative selection. \\

As a result of this thesis should be presented a method, that can be used for the representation of given clusters so that further network hosts added to the system can be correctly added to the correct cluster.
Networks monitored by this framework can have up to several tens of thousands of hosts which leads to massive clusters.
If the classification of newly added hosts were done by comparing the new host to all others, the memory and time needed for classification would be unmanageable.
That is why memory efficiency is emphasized. \\

To reach this goal general methods used for general cluster representation in non-metric spaces should be studied.
From these previously studied methods, the best ones will be selected and implemented.
As part of this thesis, there should also be created a benchmark dataset for testing the chosen method.
A further goal is to test the chosen methods on generally used clustering datasets and the created benchmark dataset.
Based on this testing, find the method that suits best for use in this field and incorporates it into the Cognitive targeted anomaly detection framework.
The selected method should be finetuned to get the best possible results in anomaly-detection.

\setsecnumdepth{all}

%========================================CHAPTER==================================

\chapter{Intrusion Detection System}

The following chapter focuses on presenting ideas from network security needed to understand the main focus of this thesis.
Different approaches of network intrusion detection systems (IDS) are explained in Section 1.
In Section 2 the scope is narrowed to network anomaly detection IDS.
Section 3 serves as an introduction to Cognitive targeted anomaly detection framework used by Cisco.
Methods researched in this thesis were tested on real-life data collected from this framework. \\ 

\section{Types of Intrusion Detection Systems}

An intrusion detection system (IDS) is a security tool designed for identification of unauthorized use or abuse of computer systems by both system insiders and external penetrators\cite{mukherjee1994network}.
IDS that is designed specifically for monitoring computer network is called Network IDS (NIDS).
Few of commonly used NIDS software is listed below:
\begin{itemize}
    \item Snort
    \item Suricata
    \item Bro Network Security Monitor
    %\item %TODO Get more examples
\end{itemize}
These systems are used on computer networks to detect and sometimes even prevent attacks that are threats to one of the essential services of such a network.
These basic services are\cite{mukherjee1994network}:
\begin{itemize}
    \item Data confidentiality
    \item Data and communication integrity
    \item Accessibility
\end{itemize}
Attackers, on the other hand, try to disrupt these services by accessing confidential information (snooping), manipulating information (man-in-the-middle attacks) or disabling access to network services (Denial of Service attacks).
IDS must have multiple components to detect as many of these attacks as is possible.
As each attack kind of intrusion is better detected by a different method, there are several types of IDS:
\begin{itemize}
    \item Signature-based (knowledge-based) %TODO put it into sentences with an explanation
    \item Anomaly-based (behavior-based)                                          
    \item Stateful protocol analysis (specification based)
\end{itemize} %TODO citation
Anomaly-based IDS serves as a foundation for work presented in this thesis.

\section{Network Anomaly Detection}                                               
Network anomaly detection is a method used in Intrusion Detection Systems (IDS) as explained in previous section.
This method focuses on comparing network host behavior changes in time.
If a change greater than a certain threshold is observed, network anomaly is detected.
In IDS when an anomaly is detected, an alert is created to inform network administrator about what has happened.
This alert can be any kind of message ranging from a syslog message to an email sent to the admin. \\ 

Host behavior is collected from devices in the network.
There is at least one (although many times multiple) device set up for collecting network flows (using NetFlow protocol).
Flow collection is the most widely used method for gathering data on the network. %TODO citation
One network flow is an aggregated information about one connection that consists of source and destination addresses, source, and destination ports, begin and end timestamps for communication and number of data in each direction. % [RFC2722] TODO citation
This aggregation of information allows much faster (even real-time) detection of problems on a network as compared to for example deep packet inspection (DPI).
DPI is another method used to detect problems in a network.
It focuses on exploring the payload of each packet.
In today's networks, this method is quickly becoming obsolete as the majority of traffic is encrypted, and therefore it is hard to perform DPI. \\ 

Network flows are sent via NetFlow protocol to one place where they are stored, and detection mechanisms are performed based on the data collected.
Network anomaly detection system creates a model of the behavior of the given network, which serves as a baseline behavior of a given network for a specific time window.
Further traffic is then collected and another model of behavior for another time window is computed.
If the behavior suddenly significantly changes it is considered to be an anomaly that is reported by the IDS. \\ %TODO talk better about time windows

\section{Challenges of Real-Time Anomaly Detection}
In theory the best way to find all all anomalies is to be able to save all traffic and calculate the model in previous section continuously on for all hosts.
This way there would be a very specific and precise model for each host in the network.
Approaching it this way, knowing all the information about the network it would be very easy to find anomaly in behaviour of each single host with a 100\% precision.
In reality this is not possible because hosts do not generate enough traffic to create a good model of their behavior.a \\

That is why models are calculated for whole networks where it is possible to create a model that reflects the behavior of the network.
This method is widely used and on smaller (several device) networks it is very efficient.
However considering a big network of thousands of devices in so many information we can easily miss one infected device generating different traffic.
The noise in the collected traffic would be too high. \\

To give an example imagine detection an infection one printer in a 10 story office building.
This printer has been infected and added to a botnet to generate traffic for DDoS (Distributed Denial of Service) attacks.
Looking at the traffic of the printer before and after infection it could easily be seen that it increased by several hundred or even thousands of percent.
If considering the traffic of the whole office building, an anomaly detection system could hardly notice a change in all the other legitimate traffic of personel working in the building. \\

%TODO Reference article - network-aware behavior clusterin of internet hosts
A way to approach this problem is explained in \cite{xu2011network}.
The model is created by creating bi-partite graphs for comunictaion of hosts in one network prefix.
These graphs are created by looking at hosts from perspective of whom does it communicate with.
From these graphs an adjacency matrix is created.
How adjacent one host is to another is calculated as a normalized weight of edges in the graph.
This adjacency matrix then serves as a baseline model for anomaly detection.
Assessing each time window of traffic means calculating an adjacency matrix for this timewindow and then comparing to the trained model.
If there is a substantional difference an anomaly is detected.
This method is proven to work, however it suffers from the noise of too many hosts explained before.
Also it might not be a good way to look at the traffic from the perspective of a network prefix, as many times a lot of different devices are on one prefix.
\\ \\
%Other ways
\\ \\
%false positives
There is also one more problem in creating a good network anomaly detection system - false positive detection.
A false positive is a network anomaly that is detected and is not an actual anomaly.
For example if we would have our anomaly detection system setup wrong it might detect an anomaly everyday at 8AM when workers come to the office, start their computers and download daily email.
Comparing the time window from before 8AM and after 8AM would not give as any relevant information.
This problem can be mitigated by setting window for which the model is calculated long enough, so that these mistakes does not happen.
In other situations false positive alerts can happen and the goal of a good network anomaly detection system is to keep them as close to 0 as possible.

\section{Cognitive Targeted Anomaly Detection Framework}

% Cognitive Threat Analytics
Algorithms studied in this thesis are tested as a part of Cognitive Targeted Anomaly Detection Framework which is a part of Cognitive Threat Analytics developed and used by Cisco.
This framework successfuly uses community-based clustering on behaviors of each host to separate whole network into smaller groups.
Running anomaly detection for each group then yields significantly better results as compared to traditional whole network approach.
Not only it works better because of commynity-based clustering, but also because of the ability to adapt dynamically as the network changes.
Thus it is able incorporate changes that happen on given network such as adding and removing devices.\\

To dive into more detail the method is separated into 2 phases.
Initial phase is used to learn the first state of the network, second ongoing phase starts when initial model is created.
During this ongoing phase the framework is able to dynamically adjust clusters to the current state of the network. \\

Initial phase starts by collecting 24 hours of traffic from given network.
Data collected consists of network traffic flows and proxy server logs.
Once the 24-hour period is over, clustering of hosts starts.
Each host is represented by a tuple $h = {S^h, F^h}$, where:
\begin{itemize}
    \item $S^h$ represents set of all visited pairs server:port
    \item $F^h$ represents the frequency of visits of server:port pairs
\end{itemize}
This frequency-based approach is chosen so that some servers like Facebook or Google do not overwhelm the statistics in the model.
When each host has its representation the community-based clustering algorithm is started using similarity measure of:
$$cos(h, h') = \frac{\sum_{s \in S} F_s^h F_s^{h'}} {\sqrt{\sum_{s \in S} (F_s^h)^2} \sqrt{\sum_{s \in S} (F_s^{h'})^2}}$$
%TODO read about and explain community based clustering
For clusters that are bigger than 30 hosts representatives are selected. %TODO check if it is really 30
Each host is then given a random number between 1 and 12 which serves as time to live for that method. %TODO check the range
This time to live is crucial for dynamic changes in the network, because of it clusters can change or even die if they do not maintain their behavior.
Currently 30 hosts are selected at random for each cluster.
In this thesis tests were made to improve this random selection method.

After the initial training phase is over and the model is established, data continues to be collected.
Every 4 hours another all hosts that are observed are assigned to existing clusters using community-based clustering explained above.
For eaach cluster number serving as time to live is decreased and hosts that reach 0 are removed from the cluster.
Free spaces are filled with hosts selected form hosts that were assigned to the cluster from this 4 hour period.
This process goes on and every 4 hours clusters are updated.

These clusters are then further used for in anomaly detection.
They serve as a foundation for calculating baseline behavior for hosts belonging to it.

%========================================CHAPTER==================================

\chapter{Theoretical Overview}
% Theoretical background to clustering and non-metric spaces

This chapter serves as a theoretical background to explain clustering methods and problematics of non-metric spaces.

\section{Introduction to Clustering}
% Briefly explain what is Clustering and how is it used in machine learning

\section{Non-Metric Space}
% jenom pairwise similarity!!! - nemuzeme delat kolecka - neplati geometricka reprezentace okoli
% centroid vubec neni v uvahu - casto nejde vubec udelat
% Louvian Modularity - sestaveni grafu na zaklade pairwise similarit - to co se pouziva v Threat - modularity clustering, community detection

\chapter{Representative Selection}
% Representative selection

\section{Approaches to Representative Selection}
% popsani jak by se to delalo normalne
% stezejni myslenky proc to delame takto

%What are the algorithms that can be used for representative selection in non-metric spaces?
%Problems that need to be tackled using this method.

\chapter{Implementation and Performance Testing of Relevant Methods}

\section{Methods Relevant to Cognitive Anomaly Detection}
The following methods were tested and compared as a part of this thesis.
\begin{itemize}
    \item Random selection
    \item Greedy Selection
    \item Delta-Medoids One-Shot
    \item Delta-Medoids %TODO citations
\end{itemize}

\subsection{Greedy Selection} %explanation of my method

\subsection{Random Selection}

\subsection{Delta-Medoids One-Shot}

\subsection{Delta-Medoids}

What criteria were chosen to compare algoritms?
Algorithms comparison.
Summary on what algorithm was chosen and why is it the best one.

\section{Datasets}
The task of cluster representation in non-metric spaces generally is not necessarily connected only to the Security field.
That is why the selected algorithms were tested on both generally know datasets like pendigits and multiple features and on specific data collected from network.
What datasets am I using and why?

\section{Testbed}

\chapter{Results}
% Testbed explanation - why is it the best etc.

What were the parameters when this method has been used?
How did it perform compared to random selection?


\setsecnumdepth{part}
\chapter{Conclusion}
Algorithms has been tested.
Which one is the best and under what circumstances?

\bibliographystyle{iso690}
\bibliography{hlavac-thesis.bib}

\setsecnumdepth{all}
\appendix

\chapter{Acronyms}
% \printglossaries
\begin{description}
	\item[GUI] Graphical user interface
	\item[XML] Extensible markup language
\end{description}


\chapter{Contents of enclosed CD}

%change appropriately

\begin{figure}
	\dirtree{%
		.1 readme.txt\DTcomment{the file with CD contents description}.
		.1 exe\DTcomment{the directory with executables}.
		.1 src\DTcomment{the directory of source codes}.
		.2 wbdcm\DTcomment{implementation sources}.
		.2 thesis\DTcomment{the directory of \LaTeX{} source codes of the thesis}.
		.1 text\DTcomment{the thesis text directory}.
		.2 thesis.pdf\DTcomment{the thesis text in PDF format}.
		.2 thesis.ps\DTcomment{the thesis text in PS format}.
	}
\end{figure}

\end{document}
